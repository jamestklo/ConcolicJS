\section{Challenges}
While an intuitive approach would be to generate DOM elements "just in time," it would not work in many cases.
In Sample Code \ref{dom0}, when {\tt getElementById()} is called, we could just create a DOM element with the corresponding id.
Yet, 
\begin {compactitem}
\item DOM tracing and constraints
\end {compactitem}




% the problem can be hard
% justify for tracing, look ahead, and generating constraints
	% greedy or just in time generation ~\cite{ZenCoding}
	% jQuery example
% justify for using a solver, rather than heuristics to resolve constraints
	% take advantage of Moore's law in terms of hardware performance and breakthroughs in constraint solvers
	% conditions example	
	% future compatibility for multiple data types

% say somewhere, each DOM operation is like a piece of a puzzle describing a subset of the HTML DOM tree.

%\header{Design Goals.}

\header{Indirect Influence.}
% drives the need for backward slicing

\header{Complex Conditions and Precedence.}
% drives the need for decorated execution, different from Jalangi

\header{Interdependent Statements.} 
% drives the need for constraints, look ahead
% JQuery ("BV"), ("div BV") Code can be collaboratively contributed by different members of team.

\header{}
% drives the need for a solver

\header{DOM mutations}

%\header{eval(), inline and native code.}
% Tudu
% Go to Implementation

%\header{Closures.}
% tinysort() as example, most of the existing frameworks for unit testing don't address this issue [phantomJS].
% drives the need for extracting functions
% Go to Implementation
