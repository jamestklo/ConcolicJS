\section{Implementation}
\label{impl}
\header{Concolic Driver.}
The concolic driver automates and coordinates the concolic process end to end from deducing constraints to generating HTML and driving execution.  Specifically, the concolic driver would repeatedly
\begin {compactitem}
\item Open the target URL in the Web browser
\item Load the generated html (initially empty html)
\item Execute the target JavaScript code
\item Measure coverage, and decide which path to go next
\item Call the MapDeducer, which returns the DOM constraints in text
\item Send the constraints to the DOM solver, which returns a satisfiable DOM tree
\item Go back to the first step with the newly generated HTML.
\end{compactitem}
The concolic driver would iterate the cycle until it has reached a certain point, which can be configured to be a specific number of iterations (e.g. 1,000 DOM trees generated), a certain level of coverage (e.g. 100\% branch coverage), or both. 


\header{Instrumenting and Executing JavaScript.}  
Our approach has to be generic, transparent and browser-independent.  
We use Selenium's WebDriver\cite{webdriverjs} to drive Web browsers for executing JavaScript.  
WebDriver runs on multiple browsers, thus using WebDriver meets our design goal of being browser-independent.  

When JavaScript is getting downloaded onto a Web browser, we use the WebScarab proxy to intercept the download and to instrumenting code.  
The proxy passes intercepted JavaScript code to the Google Closure Compiler API~\cite{ClosureCompiler} for transforming JavaScript into calls to the operator functions. 

Both the Backward Slicer and MapDeducer are implemented as JavaScript APIs, and WebDriver natively supports calling JavaScript functions within the browser.  
Thus our approach is entirely transparent and can be applied to multiple brands of Web browsers.    


\header{Inline JavaScript \& {\tt eval()}.}  
In addition to source files, JavaScript code can also be found within {\tt eval()} and inlined as attributes of a DOM element inside the HTML declaration (e.g. {\tt <body onload="runFunction()">}).
We instrument each {\tt eval(code)} statement into {\tt eval(instrument(code))}.  We cannot override the native {\tt eval()}, because the native {\tt eval()} must be called inside the closure to give {\tt code} access to the closure's local variables.  
The {\tt instrument()} function would send the {\tt code} to the proxy for instrumentation via a XMLHttpRequest.  

To instrument inline JavaScript, we traverse the webpage's original existing HTML using the JSoup API~\cite{jsoup}.  Once we detect DOM attributes that contain JavaScript, we pass the JavaScript to ClosureCompiler for instrumentation.  
For newly created elements, e.g. {\tt elem.innerHTML = text}, we use getters and setters to detect the new elements.  
Once detected, we traverse the new elements, extract JavaScript and call the {\tt instrument()} function.  
% Tudu


\header{DOM Solver.}  
CVC allows writing the constraints in Java, yet we do not want to hardcode the constraints because the constraints are different in each execution path.  
Thus we use Java's ProcessBuilder~\cite{processbuilder} class to communicate with the executable (.exe) version of CVC.  We decided to use CVC3 rather than CVC4~\cite{cvc4} because CVC3 is generally more stable during our experimentation.  
The API that parses the CVC output is also implemented in Java, thus we used the W3C DOM API~\cite{DomAPI} for building a satisfiable DOM tree and generating HTML.  
%QUnit provides a <div> called the fixture which can be set as the HTML for a test case.   Before execution, WebDriver would set the {\tt innerHTML} of the fixture element to the generated HTML.  


\header{Limited Path Coverage.}  
% zero, 1 and n.
Very often we would not know how many times to execute a loop.  For example, in Sample Code ~\ref{dom0}, there is no upper limit to the number of children that {\tt field} can have.  
\tool would execute loops zero, one, and {\tt n} times, where {\tt n} can be configured for a particular loops or for all loops.  Thus \tool would achieve limited path coverage rather than full path coverage.  


%\header{Indexing Functions.}  
%Most of the time JavaScript functions are defined inside closures and are not accessible~\cite{privatefunctions}.  
%When we instrument JavaScript code, we extract the functions by assigning them to an object that we have access to.  
%Because functions can have the same name, we use the node number in the JavaScript Abstract Syntax Tree as ID.

%\header{QUnit}
%\tool has integration with QUnit~\cite{qunit} so that existing test suites can automatically take advantage of \tool without additional manual effort.  
% \tool is also extensible to be integrated with other test frameworks~\cite{jstests}.
% Thus given a test case, be its inputs were generated manually or automatically, \tool can be used to help the test case and its assertions get fully utilized.

\header{Inserting HTML into Webpage.}
In HTML, certain DOM elements must be wrapped inside other DOM elements~\cite{jsninja}.  

For example, table cells ({\tt <td>} and {\tt <th>}) have to be contained inside a table row ({\tt <tr><td></td></tr>}).
Table rows ({\tt <tr>}) must be contained inside 
\begin{compactitem}
\item a table header: {\tt <thead> <tr></tr>... </thead>}, 
\item a table body: {\tt <tbody> <tr></tr>... </tbody>}, or
\item a table footer: {\tt <tfoot> <tr></tr>... </tfoot>}.
\end{compactitem}
\sloppy{
Similarly, every table header, table body, and table footer must be contained inside a table too: 
{\tt <table>} {\tt <thead>...</thead>} {\tt <tbody>...</tbody>} {\tt <tfoot>...</tfoot>} {\tt </table>}.
Each table contains at most 1 header, 1 body and 1 footer.  
}

{\tt <option>} and {\tt <optgroup>} have to be inside a {\tt <select multiple="multiple"> ... </select>}.
{\tt <legend>} has to be inside a {\tt <fieldset> ... </fieldset>}.
\tool processes the generated HTML before inserting it into a webpage.  
