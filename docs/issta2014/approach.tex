\section{Approach}
Given a piece of JavaScript code and a test case, our goal is to generate HTML so that the HTML's DOM structure would guide and support the execution of the code being tested towards the intended path.  
As an overview, our approach is to first instrument the JavaScript code so that we can log the code's execution for producing a dynamic backward slice and a dynamic trace.  
Next we analyze the trace and the slice to extract DOM-specific operations and deduce constraints, which the DOM solver would take as input for generating a suitable DOM structure and corresponding HTML.  
Then we integrate the HTML into a test framework for running assertions.  

%Each DOM operation in any part of the code is like a piece of a puzzle describing a subset clue of the overall DOM tree.  \tool has to systematically extract these puzzle pieces and analyse them collectively for generating a satisfiable DOM.  

\begin{figure}
\begin{lstlisting}[caption=Example showing how code is instrumented for dynamic analysis.,label=sheq]  
function beforeInstrumentation(i) {
  var row = getElementById("row"+i);
  var a = row.children.length === b; 
  if (a) {}
}

function afterInstrumentation(i) {
  var row = _CALL(getElementById, _ADD(_CONST("row"), i));
  var a = _SHEQ(_GET(_GET(row, "children"), "length"), b);
  if (__cond("IF_NAME decorated0.js 0", a)) {}
}
\end{lstlisting}
\end{figure}

\header{Decorated Execution.}
Decorated execution is where we instrument the JavaScript code so that the execution of each JavaScript operator can be captured and decorated with taint data for dynamic trace and dynamic backward slice.  Sample code \ref{sheq} illustrates the semantics of decorated execution.  
A general rule of thumb is that we transform each JavaScript operator into a function call.  For example, the {\tt .} operator is transformed into a call to {\tt \_GET()}; thus {\tt row.children} becomes {\tt \_GET(row, "children")}.  {\tt \_SHEQ} represents the strict equal operator "{\tt ===}".  
A special case happens when we transform the {\tt \&\&} and {\tt |}{\tt |} operators, in which we have to consider the precedence of the operator's left hand side.   
For example, if the code is {\tt a \&\& a.b}, the transformed version becomes {\tt \_AND(a, a.b)}; yet we don't want to execute {\tt a.b} when {\tt a} is {\tt null} or {\tt undefined}.  
Because the left hand side can be a function call, our solution is to assign the left hand side into a temporary variable, and then check the value of the temporary variable before executing the right hand side: {\tt \_AND(t = a, t \&\& a.b)}.  
Another special case is the {\tt ++} and {\tt --} operators.  For operations like {\tt i++}, we have to assign {\tt i} to a temporary variables first as return value before incrementing {\tt i}.
The base case is where constant values and inputs are defined.  If execution can be viewed in a flow, these base cases are the source, and the eventual output or return values become the sink.
Decorated execution is named after the decorator pattern~\cite{decoratorPattern}; the return value of each operator function (e.g. {\tt \_AND}) can be decorated with taint data for dynamic analysis, e.g. dynamic symbolic execution.  
% functions, boundary to native functions


\header {Trace MapReducer}
%To generate a dynamic backward slice, we must accurately capture a condition's potential nested composition and potential dynamic precedence; and decorated execution is a simple and efficient way of doing the capture.  
%Each condition is composed of 1 or more sub-conditions nested inside or linked beside other sub-conditions.  Each sub-condition is composed of 1 or more variables being compared to other variables.  


\header{Slice Analyzer}
%Decorated execution naturally captures the nested composition of conditions, i.e. when conditions are composed of sub-conditions linked by logical operators (e.g. {\tt or}'s, {\tt and}'s, {\tt not}'s) nested inside one another.
% Tree structure, Post order traversal
In a condition, dynamic backward slicing discovers what DOM operations a condition has.  
During execution, given a variable at a point in time, a dynamic backward slice traces how the variable has arrived at its current value: what operations or calculations had been previously done.  
For example, if the variable {\tt a} equals to {\tt row.children.length === b} at line 6 during execution, {\tt a}'s backward slice would be backward slices of {\tt row.children.length} and {\tt b}, linked by the strict equal {\tt ===} operator.  
Eventually a dynamic backward slice would lead us to original inputs, constant values, and environmental dependencies such as the DOM.  
For example, a dynamic backward slice of {\tt row} would lead us to the DOM element with ID {\tt "row"+i}, where {\tt "row"} is a constant string, and {\tt i} has a backward slice leading to {\tt field.children.length}, which would lead us to the DOM element with ID {\tt "field"}.


\header{DOM Solver.}
% how to create rules in DOM solver, example rules.  Intuition behind it.
% why not XML solver, not scalable
% SMT-lib language, swappable between CVC and Z3.
	% take advantage of Moore's law in terms of hardware performance and breakthroughs in constraint solvers
	% future compatibility for multiple data types

\header{Conditional Slicing.}

\header{Integration with QUnit and Selenium.}


\header{Limited Path Coverage.}


