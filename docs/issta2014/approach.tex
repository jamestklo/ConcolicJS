\section{DOM Constraints}
Given a piece of JavaScript code, and a path that we intend the code execution to follow, 
our goal is to generate a DOM tree to guide and support the code execution.  

As an overview, our approach is to first instrument the JavaScript code so that we can log the code's execution for producing a dynamic trace and a dynamic backward slice.    
Next we analyze the trace and the slice to extract DOM-specific operations and to deduce constraints, which the DOM solver would take as input for generating a satisfying DOM tree.  
The resulting DOM tree would be transformed into HTML, which we will insert into the webpage for guiding code execution.  

\begin{figure}
\begin{lstlisting}[caption=Example showing how code is instrumented for dynamic analysis.  The comment at line 9 shows the decorated object {\tt a} and its nested tree data structure.    
{\tt a}'s actual value is {\tt true} because both left and right hand side have the same value 10: {\tt line 11} and {\tt line 12},label=sheq]  
// Before Instrumentation
var row = getElementById("row"+i);
var a = row.children.length === b; 
if (a) {}

// After Instrumentation(i)
var row = _CALL(getElementById, _ADD(_CONST("STRING filename.js 0", "row"), i));
var a = _SHEQ(_GET(_GET(row, "children"), "length"), b);
/* a = {val: true
      , op:	_SHEQ
      , 0:	{val: 10, op:_GET, ...}
      , 1:	{val: 10, ...}}; */
if (__cond("IF_NAME filename.js 1", a)){}
\end{lstlisting}
\end{figure}

\header{Decorated Execution.}
Decorated execution is where we instrument the JavaScript code so that the execution of each JavaScript operator can be captured and decorated with additional data for producing a dynamic trace and a dynamic backward slice.  
Sample code \ref{sheq} illustrates the semantics of decorated execution.  
A general rule of thumb is that we transform each use of a JavaScript operator (e.g. {\tt .}) into a call to a corresponding operator function (e.g. {\tt \_GET()}).  
For example, {\tt row.children} becomes {\tt \_GET(row, "children")}.  {\tt \_SHEQ} represents the strict equal operator ({\tt ===}).  
Each operator function wraps (thus decorates) the actual computed value inside a decorated object that also contains other data for tracing and slicing.   

A special case happens when we transform the {\tt \&\&} and {\tt |}{\tt |} ({\tt or}) operators, in which we have to consider the precedence of the operator's left hand side.   
For example, if the code is ({\tt a \&\& a.b}), the transformed version becomes {\tt \_AND(a, a.b)}; yet we do not want to execute {\tt a.b} when {\tt a} is {\tt null} or {\tt undefined}.  
A possible solution is to reuse {\tt a}: {\tt \_AND(a, a \&\& \_GET(a, "b")}.  
However, the left hand side can be a call to a function that may change the internal state of the application: e.g. {\tt appendLog() \&\& update()}.
Thus our solution is to assign the left hand side into a temporary variable, and then check the value of the temporary variable before executing the right hand side: 
{\tt \_AND(t = a, t \&\& \_GET(a, "b"))} and {\tt \_AND(t = \_CALL(appendLog), t \&\& \_CALL(update))}.  

Another special case is the {\tt ++} and {\tt ---} operators.  
For example, with {\tt i++} we have to first assign the original value of {\tt i} to a temporary variable before incrementing {\tt i}, then we return the temporary variable.
% functions, boundary to native functions


\header{Backward Slicer \& Post Order Traversal.}
% would a diagram be good?
The data structure of the decorated objects can be seen as a nested or tree structure because the calls to the operator functions are nested inside one another.  
For example, in Sample Code \ref{sheq}, the call to {\tt \_GET(..., "length")} is nested inside the call to {\tt \_SHEQ()}.  
Therefore, if we simply put the name of the operator function (e.g. {\tt "\_GET"}, {\tt "\_SHEQ"}, ...), inside the trace data, we can easily generate a backward slice via a tree traversal.  

In Sample Code \ref{sheq}, the variable {\tt a} equals to ({\tt row.children.length === b}).  
Thus {\tt a}'s backward slice must contain the backward slice of {\tt b} and the backward slice of {\tt row.children.length}, linked by the strict equal ({\tt ===}) operator.  
At line 8, the decorated object returned by {\tt \_SHEQ()}, assigned to the variable {\tt a}, is the tree parent of 2 decorated objects: {\tt b}, and the decorated object returned by the {\tt \_GET()} call.  

The tree children of a decorated object always come from earlier executions, e.g. {\tt \_GET(..., "children")} is executed before {\tt \_GET(..., "length")} before {\tt \_SHEQ(..., b)}.
Thus the tree's hierarchical structure is reversely proportional to the temporal order in which the operator functions are executed.  

During the traversal, we identify conditions that contain DOM operations and extract these DOM operations accordingly.  
In a chain of DOM operations, the operations closer to the chain head always come from earlier executions, thus the tree's hierarchy is also reversely proportional to the chaining order of DOM operations.  
The backward slicer traverses the decorated objects in post order, which is bottom up from leaf to root.  
This way, the dynamic backward slice not only yields a temporal history of the code's runtime execution, it also conveniently traces the DOM operation chains in the order from head to tail.

Each tree leaf represents an input or a constant.  
For example, a dynamic backward slice of {\tt row} would lead us to the DOM element with ID {\tt "row"+i}, where {\tt "row"} is a constant string, 
and {\tt i} has a backward slice leading to {\tt field.children.length}, which would lead us to the DOM element with ID {\tt "field"}.  
Because variables can be used multiple times, each variable can belong to more than 1 tree and can have more than 1 parent.  
Thus the data structure would appear more like a directed acyclic graph than a tree, even though a variable would never be a tree ancestor of any of its own ancestors.  

\begin{figure}
\begin{lstlisting}[caption=Constraints for generating a DOM tree that would satisfy for going the {\tt True} branch in the {\tt if} statement of Sample Code ~\ref{dom0}.  The constraints are shown in the input format for the CVC~\cite{cvc3} implementation of the SMT solver. {\tt \%} is the comment operator in CVC.,label=constraints0]
% document.getElementById("field");
% document.getElementById("row"+0);
ASSERT DISTINCT(field, row0);

% (field.children.length)--;
ASSERT childrenLength(field) > 0;

% row.children.length === 10;
ASSERT childrenLength(row0) = 10;
\end{lstlisting}
\end{figure}

\header {Trace Mapper \& Constraints Deducer.} 
For each instance that a condition is executed, the backward slicer would yield what DOM operations the instance has and how these DOM operations are related or interdependent on one another.  
Because each condition can get executed more than once at different time points, the MapDeducer would aggregate all executed conditions, map them according to their ID, and deduce constraints for the DOM solver to generate a satisfiable DOM tree.  
So far everything is code-oriented in which we focus on each condition and its dynamic backward slice.  The MapDeducer would transition the focus to be DOM-oriented in which we assemble clues about the same part of the DOM tree that are scattered across multiple lines of code.  
The MapDeducer would put together the processed clues across multiple parts of the DOM tree back together, into a single set of constraints for the DOM solver to generate a satisfiable DOM tree.

Sample Code~\ref{constraints0} illustrates constraints for going to the {\tt true} branch of the {\tt if} statement in Sample Code~\ref{dom0}, resulting in Sample Code ~\ref{html0}.  
If we want to go to the {\tt false} branch, e.g. {\tt ASSERT NOT(childrenLength(row0) = 10)}, then the solver would generate a number of children not equal to 10 for {\tt row0}.  The exact number of children has not been deterministic based on our experiments: sometimes {\tt row0} has 2 children, sometimes {\tt row0} has none.  
%Recall the jQuery example from the Challenges section.  The MapDeducer would 

\begin{figure}
\begin{lstlisting}[caption=Example HTML generated from the results of the DOM solver based on the constraints defined in Sample Code ~\ref{constraints0}.  Note that {\tt row0} is not a child of {\tt field} because the source code in Sample Code \ref{dom0} did not require the rows to be children of {\tt field}.,label=html0]  
<span id="field"><span></span></span>
<span id="row0">
  <span></span><span></span>
  <span></span><span></span>
  <span></span><span></span>
  <span></span><span></span>
  <span></span><span></span>
</span>
\end{lstlisting}
\end{figure}
