\section{Introduction}

JavaScript is increasingly an attractive programming language for implementing software applications.
The World Wide Web is increasingly a popular platform of delivering them.
HTML5, BYOD (Bring Your Own Device), Node.js
http://www.microsoft.com/en-us/news/press/2013/nov13/11-06bnsfcustomerspotlightpr.aspx

Unfortunately, testing JavaScript code is not trivial.  
A major reason is that a lot of JavaScript code is intermixed with back end logic and front end HTML.  
This paper focuses on the problem of generating HTML for testing JavaScript code.
\cite{QUnitIntro}
[http://qunitjs.com/intro/] 
manual generation
random generation
greedy or just in time generation




Random testing, concolic testing... many prior work focus on generating inputs. 
However, the problem is that just having the inputs to a function is not sufficient.  
For example, DOMtris is a very simple Web implementation of Tetris, almost all functions don't have any input arguments.  
Yet, all these functions both depend on and mutate the state of the app.  
In the case of DOMtris, an example use case is to increase score and clean up the game pieces when the gamer has fully filled one or more rows.  
The DOM makes up a big part of a Web's app state.  
Thus in this paper we focus on the DOM.  
Indeed, DOM operations make up a majority of an app's execution and defects, according to a study by Frolin.  

A naive approach is to do just in time generation of the DOM.  
For example, if the statement is document.getElementById(), we can just return or create an element with "id".  
Yet there are 2 subtle issues.

First, there are multiple possible outputs.  

Second, dependencies in Wikipedia.  

Using a solver is easier and protects us from more complicated cases.

Our approach also supports concolic testing of DOM operations.



The new perspective is distinct from our previous motivation, which was using our tool to define test cases and drive run time execution of the program.  

Inputs can be defined for a test case in different ways: manually by test engineers writing test cases or automatically through concolic or random testing
