\section{DOM Solver}
The DOM solver takes the constraints defined by the MapDeducer and attempts to generate a satisfiable DOM structure.  The solver is implemented as an extension of a SMT solver~\cite{cvc3} and would report anything not satisfiable.  

\header{DOM tree \& DOM operations.}
A major part of the DOM is its multi-children tree structure.  When generating a satisfiable DOM, we use the DOM operations executed to infer the overall DOM tree.    
Each DOM operation in any line of code is like a piece of a puzzle describing a subset clue of the overall DOM tree.   
For example {\tt a = elem.parentElement.nextElementSibling} implies 2 subset clues: {\tt elem} has a parent element, and the parent has a sibling.

Yet, questions remain unanswered about exactly where does {\tt elem} fit in the overall DOM tree; and other DOM operations would provide clues for that.  The DOM solver would take all the clues and generate a satisfiable structure.   

\begin{figure}
\begin{lstlisting}[caption=Example extended from Sample Code ~\ref{domOr} to demonstrate capability of DOM solver to resolve more complex DOM operations and their interdependency with each other.,label=domOrExtended]  
// Interdependent logical constraints
if (d === elem.firstElementChild // i)
 || d === b.lastElementChild) {}
if (d === elem.parentElement // ii)
 || d === b.parentElement) {}
  
if (b.previousElementSibling === 
    c.firstElementChild) {}
  
if (elem.parentElement.parentElement 
    === c.lastElementChild.previousElementSibling) {}
	  
if (c.children[2] === 
    b.nextElementSibling) {}  
}
\end{lstlisting}
\end{figure}


\begin{figure}
\begin{lstlisting}[caption=Condensed example extended from Sample Code ~\ref{domOr} demonstrating various DOM operations and their interdependency with each other.,label=domOrExtended]  
<span id="c">
  <span/>
  <span id="b">
    <span id="d">
      <span id="elem"/>
    </span>
  </span>
  <span/>
</span>
\end{lstlisting}
\end{figure}


% XPath
% how to create rules in DOM solver, example rules.  Intuition behind it.
% why not XML solver, not scalable
% SMT-lib language, swappable between CVC and Z3.
	% take advantage of Moore's law in terms of hardware performance and breakthroughs in constraint solvers
	% future compatibility for multiple data types

\header{DOM operations into Boolean functions.}	


\header{Solver Output into Heuristics.}
Because CVC does not support strings natively, for now we map tag names into integers in CVC.  
<span> is the default tag type because <span> is an inline element and not a block.


\header{Ordering Children.}


\header{DOM mutations.}


\header{Conditional Slicing.}

