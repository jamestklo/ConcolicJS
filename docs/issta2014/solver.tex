\section{DOM Solver}
The DOM solver takes the constraints defined by the MapDeducer and attempts to generate a satisfiable DOM structure.  The solver is implemented as an extension of a SMT solver~\cite{cvc3} and would report anything not satisfiable.  

\begin{figure}
\begin{lstlisting}[caption=Example extended from Sample Code ~\ref{domOr} with more complex DOM operations and their interdependency with each other.  The added {\tt if} statements influence the solver to decide on {\tt i)} or {\tt ii)},label=domOrExtended]  
// Interdependent logical constraints
if (d === elem.firstElementChild // i)
 || d === b.lastElementChild) {}
if (d === elem.parentElement // ii)
 || d === b.parentElement) {}

if (b.previousElementSibling === 
    c.firstElementChild) {}

// chained DOM operations 
if (elem.parentElement.parentElement 
    === c.lastElementChild.previousElementSibling) {}  
if (c.children[2] === 
    b.nextElementSibling) {}  
}
\end{lstlisting}
\end{figure}

\header{DOM tree \& DOM operations.}
A major part of the DOM is its single parent, multi-children tree structure.  When generating a satisfiable DOM, we use the execution of DOM operations to infer the overall DOM tree.    
Each DOM operation in any line of code is like a piece of a puzzle describing a subset clue of the overall DOM tree.   
For example {\tt a = elem.parentElement.nextElementSibling} implies 2 subset clues: {\tt elem} has a parent element, and the parent has a sibling.  
Note that if the condition is {\tt a = elem.parentElement.nextElementSibling === null}, then the clues become {\tt elem} has a parent element, yet the parent has no next sibling and thus is the last child.

That said, questions remain unanswered about exactly where does {\tt elem} fit in or belong in the overall DOM tree; and other DOM operations would provide clues for that.  The DOM solver would take all the clues and generate a satisfiable structure.   

\begin{figure}
\begin{lstlisting}[caption=HTML generated for going the all {\tt true} path in Sample Code ~\ref{domOrExtended}.  The all {\tt true} path is the execution path in which every condition goes to the {\tt true} branch.,label=htmlExtended]  
<span id="c">
  <span/>
  <span id="b">
    <span id="d">
      <span id="elem"/>
    </span>
  </span>
  <span/>
</span>
\end{lstlisting}
\end{figure}


% XPath
% how to create rules in DOM solver, example rules.  Intuition behind it.
% why not XML solver, not scalable
% SMT-lib language, swappable between CVC and Z3.
	% take advantage of Moore's law in terms of hardware performance and breakthroughs in constraint solvers
	% future compatibility for multiple data types
	
\header{DOM operations into SMT functions.}	
Each DOM operation is transformed into a SMT function in the solver.  We then use quantifiers (e.g. {\tt EXISTS}, {\tt FORALL}) to define how the SMT functions relate to each other.  
Sample Code ~\ref{childrenLength} shows the boolean functions and integer functions we defined for supporting the {\tt elem.children.length} operation.  We first quantify the parent-child relationship: 
\begin{compactitem} 
\item a node cannot be a child of itself, see {\tt i)} and {\tt ii)} in Sample Code ~\ref{childrenLength}.
\item a child of a node cannot be the node's parent at the same time: {\tt iii)}.
\item a child can have only 1 parent: {\tt iv)}.
\end{compactitem}
Next, we define how children are ordered and quantify {\tt children.length}:
\begin{compactitem} 
\item first child starts at position or index 0: {\tt v)} and {\tt vi)}.
\item last child has the largest child index: {\tt vii)}.
\item {\tt children.length} equals to the child index of the last child: {\tt viii)}.
\end{compactitem}
The parent SMT functions are quantified as the inverse of the child SMT functions (e.g. {\tt ix}).  Similar to {\tt firstChild()} and {\tt lastChild()}, the sibling SMT functions are defined by extending {\tt children(x, y, j)}.  
For example, the next sibling of a node has the same parent and child index {\tt j+1}, when the node has child index {\tt j}.

\begin{figure}
\begin{lstlisting}[caption=SMT functions for defining the children.length DOM operation.  We start with defining the parent-child relationships; then move on to the ordering of children; then use the child index of the last child to define and quantify the {\tt childrenLength()} boolean function,label=childrenLength]  
% i) child(x, y): x is a child of y.
child: (Node, Node) -> BOOLEAN;	

% ii) x cannot be a child of itself
ASSERT FORALL (x: Node):	
  NOT(child(x,x));
	
% iii) when y is the parent of x
% then y cannot be a child of x.  
ASSERT FORALL (x, y: Node):
  child(x,y) => NOT(child(y,x));
  
% iv) a child has only 1 parent
ASSERT FORALL (x, y, z: Node):
  (child(x,y) AND DISTINCT(y,z)) => NOT(child(x,z));

  
% v) x is the j-th child of y 
children: (Node, Node, INT) -> BOOLEAN;
ASSERT FORALL (x,y:Node, j:INT): 
  children(x, y, j) => 
    child(x, y) AND j >= 0;

% vi) child index starts at 0.
firstChild:		(Node, Node) -> BOOLEAN;
ASSERT FORALL (x, y:Node):
  firstChild(x, y) <=> 
    children(x, y, 0);

% vii) no other child has a child index 
% larger than the one of the last child
lastChild: (Node, Node) -> BOOLEAN;	
ASSERT FORALL (x, y:Node): 	
  lastChild(x, y) => EXISTS(j:INT): 
   ( j >= 0 AND children(x, y, j) AND 
      (NOT(EXISTS (z: Node, k:INT): 
	    children(z, y, k) AND k > j)) );

% viii) children.length equals to the
% child index of the last child.
childrenLength:	(Node) -> INT;
ASSERT FORALL (y:Node, j:INT):
	childrenLength(y) = j <=> EXISTS(x:Node): lastChild(x, y) AND children(x, y, j);

% ix) example of inversion
% y is the parent of x, is the same as 
% x is a child of y
parent: (Node, Node) -> BOOLEAN; 
ASSERT FORALL (x, y: Node): 
  parent(y, x) <=> child(y, x); 
\end{lstlisting} 
\end{figure}

\header{Negation \& XPath.}
We are using the CVC implementation of SMT, which natively supports booleans and integers by default.  
While it is possible to define a customized data structure to mimic the DOM tree in CVC, the boolean/integer function approach is simple and convenient for negating conditions and going into different execution paths.
Because Web browsers also support accessing the DOM tree via XPath~\cite{documentEvaluate}, we defined additional SMT functions: {\tt following\_sibling}, {\tt preceding\_sibling}, {\tt descendant} and {\tt ancestor}.

\header{Solver Output into HTML.}
Because we used Boolean functions and Integer functions, CVC is only going to solve for the interdependent logic constraints and it does not directly yield a concrete DOM tree.  
Instead, CVC only expands the quantifiers and outputs more {\tt ASSERT} statements that is valid for making the constraints satisfiable.  

In Sample Code \ref{domOrExtended}, to solve ({\tt d === elem.firstElementChild || d === b.lastElementChild}), CVC would output "{\tt ASSERT lastChild(d, b)}" because it has decided on making ({\tt d === b.lastElementChild}) {\tt true}.

To build the DOM tree, we built an API that parses the CVC output and builds a model for the DOM.  
In the CVC output, CVC creates many temporary variable names, thus each DOM element can have more than one alias.  
To consolidate the aliases, we start with DOM operations expressing the parent child relationship because each child can have only 1 parent.  We also used other deterministic DOM operations such as {\tt firstChild()} and {\tt lastChild()} to group aliases together.  

Once the parent child relationship is built, our next step is to organizer the ordering of children.  
Some DOM children have their positions explicit (e.g. {\tt firstChild(x, y)} and {\tt children(x, y, j)}), yet very often the child parent relationship is implied.  For example, {\tt nextSibling(x, z)} implies {\tt x} and {\tt z} share the same parent.  
To calculate the ordering of children, we use the explicitly positioned children as anchors and we relate the anchors to other children by the sibling operations.  

In case of any uncertainty, we would query the CVC model.  We have to do a binary search because querying the CVC model supports only {\tt true} or {\tt false} answers.  
Because CVC does not support strings natively, for now we map tag names into integers in CVC.  <span> is the default tag type because <span> is an inline element and not a block.
Once the DOM tree is defined, we search for the root of the DOM tree (the element without a parent) and generate HTML.  

% \header{DOM mutations.}

% \header{Conditional Slicing.}
